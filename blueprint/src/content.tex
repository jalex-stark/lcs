% Blueprint content for Robust LCS Self-Testing
% This file contains the mathematical content linked to the Lean 4 formalization

\chapter{General self-testing}\label{sec:self-testing}

In this section, we introduce our main robust self-testing theorem for linear constraint system games with solution groups of a certain form. We start by stating and proving an exact version of the theorem, then introduce the necessary tools to prove an approximate version.

\section{Exact self-testing}
\label{subsection:exact-self-testing}

Throughout, let $\text{LCS}(\mathbf{H}, l, \mathbb{Z}_d)$, $\mathbf{H} = (V,E,H)$ be an LCS game with solution group $\Gamma$.

\begin{theorem}[Rigid self-testing of observables]\label{thm:rigid-self-testing-observables}
\lean{RobustLCS.Exact.ExactSelfTest.rigidSelfTest}
\uses{lem:unique-operator-solution, lemma:CLS}
Suppose $\Gamma$ is finite and all of its irreducible representations with $J\mapsto \omega_d I$ are equivalent to a fixed irrep $\sigma:\Gamma\to U(\mathbb{C}^d)$. Suppose $\{A_e^{(v)}\}, \{B_e\}, \rho\in \mathcal{L}(\mathcal{H}_A\otimes \mathcal{H}_B)$ is a perfect strategy presented via observables for the game.
Then there are local isometries $V_A, V_B$ such that
	\begin{itemize}
		\item for all $e,v$, $V_AA_e^{(v)}V_A^\dagger = \sigma(e)\otimes I \oplus \hat{A}^{(v)}_e$, where $\hat{A}^{(v)}_e V_A\rho V_A^\dagger = 0$, and
		\item for all $e$ $V_BB_eV_B^\dagger =\bar{\sigma(e)}\otimes I \oplus \hat{B}_e$, where $\hat{B}_e V_B\rho V_B^\dagger = 0$.
	\end{itemize}
\end{theorem}

\section{State-dependent distance}\label{subsection:state-dependent-distance}

We now begin to collect the necessary tools to generalize the previous section to the approximate case. To start, we need a convenient calculus for manipulating our notion of state-dependent distance.

\begin{definition}[State-dependent distance]\label{def:state-distance}
\lean{RobustLCS.Core.Density.Drho}
\uses{def:density-matrix}
For a density matrix $\rho$ and operators $X, Y$, we define
\begin{equation}
D_\rho(X \| Y) = \sqrt{\mathrm{Tr}_\rho(X-Y)^\dagger(X-Y)}
\end{equation}
\end{definition}

\begin{definition}[Density matrix]\label{def:density-matrix}
\lean{RobustLCS.Core.Density.Density}
A density matrix $\rho$ on $\mathcal{H}$ is a positive semidefinite Hermitian operator with trace 1.
\end{definition}

Much like the fidelity of quantum states, the squared distance $D_\rho(\cdot \| \cdot)^2$ is often more natural than the distance. We collect computationally useful properties of $D_\rho$ in the following lemma.

\begin{lemma}[State-dependent distance properties]\label{lemma:state-dependent-distance}
\lean{RobustLCS.Core.StateDistance}
\uses{def:state-distance, def:density-matrix}
	Let $\mathcal{H} = \mathcal{H}_A\otimes \mathcal{H}_B$ be a Hilbert space. Let $U,U_i$ be unitary operators on $\mathcal{H}$. Let $Z,Z_i$ be arbitrary operators on $\mathcal{H}$. Similarly, let $A_i, B_i$ be unitary operators on $\mathcal{H}_A, \mathcal{H}_B$ respectively. Let $X_i, Y_i$ be arbitrary operators on $\mathcal{H}_A, \mathcal{H}_B$, respectively.
	Let $\rho$ be a state on $\mathcal{H}_A\otimes \mathcal{H}_B$. Let $V: \mathcal{H}\to \mathcal{H}'$ be an isometry and $U'$ a unitary operator on $\mathcal{H}'$. Then
	\begin{enumerate}[(a)]
		\item
		\label{item:state-dependent-distance-square}
		\lean{RobustLCS.Core.StateDistance.Drho_sq_formula}
		$D_\rho(U \| I)^2 =2 - 2\mathrm{Re}\,\mathrm{Tr}_\rho U$. More generally, $D_\rho(Z \| I)^2 = 1 + \mathrm{Tr}_\rho Z^\dagger Z - 2\mathrm{Re}\,\mathrm{Tr}_\rho Z$.
		\item
		\label{item:state-dependent-distance-inverse}
		\lean{RobustLCS.Core.StateDistance.Drho_left_unitary}
		$D_\rho(UX \| UY) = D_\rho(X \| Y)$ for unitary $U$. (Left-unitary invariance)
		\item\label{item:state-dependent-distance-triangle}
		\lean{RobustLCS.Core.StateDistance.Drho_triangle}
		% \leanok
		$D_\rho(Z_1 \| Z_3) \leq D_\rho(Z_1 \| Z_2) + D_\rho(Z_2 \| Z_3)$. (Triangle inequality)
		\item\label{item:state-dependent-distance-right-multiplication}
		\lean{RobustLCS.Core.StateDistance.Drho_left_mul_bound}
		$D_\rho(ZU_2 \| U_3) \leq D_\rho(Z \| I) + D_\rho(U_2 \| U_3)$ for unitaries $U_2, U_3$.
		\item\label{item:state-dependent-distance-chaining}
		\lean{RobustLCS.Core.StateDistance.Drho_chain_sum}
		$D_\rho(\prod_{i}A_i\otimes I_B \| \prod_{i}I_A \otimes B_i) \leq \sum_i D_\rho(A_i\otimes I_B \| I_A\otimes B_i)$. (Chain sum bound)
		\item\label{item:state-dependent-distance-conjugation}
		\lean{RobustLCS.Core.StateDistance.Drho_unitary_push}
		If $D_\rho(I_A\otimes WB \| I) \leq \nu$ and $D_\rho(A\otimes B \| I)\leq \eta$, then $D_\rho(I_A\otimes BW \| I)\leq \nu + 2\eta$ for unitary $W$. (Unitary push)
		\item\label{item:state-dependent-distance-jensen}
		\lean{RobustLCS.Core.StateDistance.Drho_convexity}
		$D_\rho(\mathbb{E}_i U_i \| I) \leq \mathbb{E}_i D_\rho(U_i \| I)$. (Convexity)
		\item \label{item:state-dependent-distance-partial-trace}
		\lean{RobustLCS.Core.StateDistance.Drho_tensor_I_eq_marginal}
		$D_\rho(A\otimes I_B \| I_{AB}) = D_{\rho_A}(A \| I_A)$, where $\rho_A = \mathrm{Tr}_B \rho$. (Partial trace)
		\item \label{item:state-dependent-distance-isometry}
		\lean{RobustLCS.Core.StateDistance.Drho_isometry_covariant}
		$D_\rho(Z_1 \| Z_2) = D_{V\rho V^\dagger}(VZ_1V^\dagger \| VZ_2V^\dagger)$ for isometry $V$. (Isometry covariance)
		\item\label{item:state-dependent-distance-projection-is-identity}
		\lean{RobustLCS.Core.StateDistance.Drho_proj_support}
		If $P$ is a projection such that $P\rho = \rho$, then $D_\rho(XP \| I) = D_\rho(X \| I) = D_\rho(X \| P)$. (Projection support)
	\end{enumerate}
\end{lemma}

\begin{proof}
\uses{lemma:state-dependent-distance}
The proofs are routine matrix calculations using the definition of $D_\rho$ and basic properties of the trace. See the Lean formalization for complete details.
\end{proof}

We'll use properties (f) and (d) to convert proofs of group relations into proofs of approximate relations between operators which try to represent the group.

\section{Robust self-testing (future work)}

The remainder of the self-testing machinery---including the stability lemma, quantitative van Kampen, and the main robust self-testing theorem---will be formalized in later phases of this project.
